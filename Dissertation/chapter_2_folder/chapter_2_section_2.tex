\section{Формирование тестовых наборов данных} \label{sec:test_data_create}

Общая схема формирования нелинейных искажений в сигнале передатчика представлена в разделе~\ref{sec:testbench_descr}. В данном разделе рассматриваются схемы формирования данных под конкретные задачи компенсации искажений в устройствах связи.

\subsection{Формирование обучающей выборки для задачи компенсации нелинейной помехи в приёмнике полнодуплексной системы связи} \label{subsec:data_ibfd}

Полнодуплексная система связи подразумевает передачу и приём данных в одной и той же полосе частот. Поскольку в реальных устройствах связи существуют проблемы в обеспечении изоляции передатчика и приёмника, сигнал передатчика высокой мощности попадает в канал приёмника и ухудшает качество приёма.

Будем считать, что канал распространения помехи передатчик-приёмник линейный. Частотная характеристика этого канала сформирована при помощи программного обеспечения Python в соответствии с данными о реальном канале распространения помехи и представлена на рис.~\ref{fig:channel_fir}

Для симуляции прохождения нелинейной помехи по различным путям распространения от передатчика к приёмнику производится свёртка массива отсчётов $\text{PA}(\bit{x})$ с коэффициентами $\bit{w}$ импульсной характеристики канала:
\begin{equation}
	\bit{d}=\text{conv}_{\bit{w}}\text{PA}(\bit{x})
	\label{rx_noise_generate}
\end{equation}

Таким образом, получаем сигнал \bit{d}, имитирующий сигнал паразитной помехи передатчика на приёмнике мобильного терминала. Спектральная плотность мощности сигнала паразитной помехи~\bit{d} изображена на рис.~\ref{fig:rx}. 

\begin{figure}
	\begin{subfigure}[b]{0.51\textwidth}
		\includegraphics[scale=0.42]{figures/data_for_thesis/IBFD_single_band/channel_fir.pdf}
		\caption{}
		\label{fig:channel_fir}
	\end{subfigure}
	\hspace{2ex}
	\begin{subfigure}[b]{0.51\textwidth}
		\includegraphics[scale=0.42]{figures/data_for_thesis/IBFD_single_band/rx.pdf}
		\caption{}
		\label{fig:rx}
	\end{subfigure}
	\caption{(\subref{fig:tx}) Амплитудно-частотная характеристика канала распространения второй гармоники от передатчика к приёмнику; (\subref{fig:pa_out}) паразитная помеха на приёмнике \bit{d}}
	\label{fig:firfir_example}
\end{figure}

\subsection{Формирование обучающей выборки для задачи компенсации второй гармоники в приёмнике системы связи} \label{subsec:data_2nd_harmonic}

Ввиду сложности обеспечения изоляции дуплексного фильтра приемо-передатчики, работающие в режиме частотного разделения каналов, подвержены утечке передаваемого сигнала в приемный тракт. Утечка сигнала в сочетании с нелинейностью малошумящего усилителя (МШУ) и смесителя понижения частоты приемника может привести к генерации интермодуляционных искажений второго порядка, что значительно снижает чувствительность приемника на частоте базовой полосы.

Схема генерации нелинейных искажений второго порядка представлена на рис.~\ref{fig:testbench_imd2}.
\begin{figure}[h!]
	\centering
	\includegraphics[scale=1.0]{figures/testbench/testbench_imd2.pdf}
	\caption{Установка формирования нелинейных искажений второго порядка в приёмнике устройства связи.}
	\label{fig:testbench_imd2}
\end{figure}
Экспериментальная установка состоит из компьютера (ПК), усилителя мощности (УМ) ZRL-3500+. Выход УМ подключен к полосовому фильтру, который представляет собой дуплексный фильтр (ДФ) с подавлением 30 дБ в полосе заграждения. Выход ДФ подключен к малошумящему усилителю (МШУ) ZRL-3500+ с коэффициентом усиления 26 дБ. Для преобразования частоты использовался частотный смеситель ZX05-63LH-S+. В передающей и приемной частях дополнительные аналоговые фильтры не использовались.

Для экспериментов использовался комплексный сигнал OFDM с шириной полосы 5 МГц, $f_{Tx} = 814$ МГц, $f_{Rx} = 859$ МГц, дуплексное разделение 45 МГц (5G NR Band 26). Сигнал LTE передается генератором сигналов R\&S SMW 200A и усиливается усилителем мощности (УМ). Передаваемый сигнал после нелинейного УМ проникает через полосу заграждения дуплексера с частотным сдвигом 45 МГц относительно сигнала гетеродина и усиливается МШУ. Усиленный сигнал генерирует интермодуляционные помехи на выходе частотного смесителя. После понижения частоты сигнал захватывается цифровым осциллографом DSO9254A. Сигнал гетеродина мощностью 10 дБм для ZX05-63LH-S+ формируется генератором сигналов R\&S SMW 200A.

Передаваемая мощность на выходе УМ установлена на уровне $P_{Tx} = 8$ дБм, что в сочетании с аттенюацией дуплексного фильтра 30 дБ (при $f_{Tx}=814$ МГц) и усилением МШУ 26 дБ обеспечивает мощность паразитного сигнала на входе смесителя понижения частоты $P=8$ дБм $-$ 30 дБ $+$ 26 дБ $\approx$ 4~дБм.

Сигнал передатчика $\bit{x}$ на нулевой частоте, а также сигнал на выходе нелинейного УМ $\text{PA}(\bit{x})$ изображены на рис.~\ref{fig:tx_rx_imd2}
\begin{figure}
	\begin{subfigure}[b]{0.51\textwidth}
		\includegraphics[scale=0.42]{figures/data_for_thesis/SIC_second_harmonic_single_band/tx.pdf}
		\caption{}
		\label{fig:tx_imd2}
	\end{subfigure}
	\hspace{2ex}
	\begin{subfigure}[b]{0.51\textwidth}
		\includegraphics[scale=0.42]{figures/data_for_thesis/SIC_second_harmonic_single_band/rx.pdf}
		\caption{}
		\label{fig:rx_imd2}
	\end{subfigure}
	\caption{Спектральная плотность мощности: (\subref{fig:tx}) сигнала передатчика до прохождения нелинейных компонент; (\subref{fig:pa_out}) на выходе нелинейного УМ $\text{PA}(\bit{x})$}
	\label{fig:tx_rx_imd2}
\end{figure}. Заметим, что нелинейное искажение второго порядка $\bit{x}^2$ формируется засчет нелинейности компонент смесителя. Этот процесс можно формально разделить на 2 шага: формирование 2-ой гармоники $\bit{x}^2$, а затем перенос на несущую частоту передатчика $\bit{x}^2e^{j2\pi f_{Tx}t}$. \textbf{Уточнить процесс формирования второй гармоники.}

\subsection{Формирование обучающей выборки для задачи цифровых предыскажений в передатчике двухканальной системы связи} \label{subsec:data_dpd_2channel}

Система цифрового предыскажения для двухканальной системы связи изображена на рис.~\ref{fig:testbench_2ch}.
\begin{figure}[h!]
	\centering
	\includegraphics[scale=0.6]{figures/testbench/testbench_2ch.pdf}
	\caption{Схема компенсации нелинейных искажений в передатчике двухканальной системы связи.}
	\label{fig:testbench_2ch}
\end{figure}
Цифровые сигналы каналов A и B на нулевой частоте: $\bit{x}_A$ и $\bit{x}_B$ соответственно, - проходят через цифровой повышающий преобразователь (ЦПП), преобразовываются в аналоговый сигнал. Затем суммарный двухканальный сигнал переносится на несущаю частоту $f_{гет}=1990$ МГц. Частотный разнос канаов составяет 300 МГц. Таким образом сигналы каналов A и B находятся на несущих 1840 МГц и 2140 МГц соответственно~\cite{NCC2020IS2051}. Спектральная плотность мощности суммарного узкополосного сигнала передатчика на выходе УМ изображена на рис.~\ref{fig:tx_dpd_x_2ch}. 
\begin{figure}[h!]
	\centering
	\includegraphics[scale=0.5]{figures/data_for_thesis/DPD_double_band/x.pdf}
	\caption{Спектральная плотность мощности двухканального узкополосного сигнала передатчика до прохождения нелинейного УМ.}
	\label{fig:tx_dpd_x_2ch}
\end{figure}

Суммарный узкополосный сигнал далее проходит через нелинейный УМ и отправляется в петлю обратной связи через направленный ответвитель. Спектральная плотность мощности узкополосного сигнала на выходе УМ изображена на рис.~\ref{fig:rx_dpd_x_2ch}.
\begin{figure}[h!]
	\centering
	\includegraphics[scale=0.5]{figures/data_for_thesis/DPD_double_band/d.pdf}
	\caption{Спектральная плотность мощности двухканального узкополосного сигнала передатчика после прохождения нелинейного УМ.}
	\label{fig:rx_dpd_x_2ch}
\end{figure}

В петле обратной связи узкополосный сигнал переносится на нулевую частоту, оцифровывается и проходит через цифровой понижающий преобразователь (ЦПП). ЦПП разделяет каналы и переносит их на нулевую частоту. В результате получаем 2 цифровых нелинейно искаженных сигнала передатчика $\bit{d}_A$, $\bit{d}_B$ на нулевой частоте.

В разделе~\ref{subsec:dpd_describtion} главы~\ref{chapter:intro} показано, что задача компенсации нелинейных искажений в передатчике сводится к идентификации нелинейных искажений на основе исходного сигнала передатчика~\cite{dpd_thesis}. Таким образом, обучающая выборка состоит из исходных сигналов передатчика на нулевой частоте $\bit{x}_{A}$, $\bit{x}_{B}$, а также сигналов отклонения выхода УМ от входа:
\begin{align}
	\bit{e}_A&=\bit{d}_{A}-\bit{x}_{A} \nonumber \\
	\bit{e}_B&=\bit{d}_{B}-\bit{x}_{B}. \nonumber
	\label{dpd_2ch_targets}
\end{align}
Спектральные плотности мощности нелинейных искажений на нулевой частоте $\bit{e}_A$, $\bit{e}_B$ изображены на рис.~\ref{fig:error_dpd_2ch}.

\begin{figure} [h!]
	\begin{subfigure}[b]{0.51\textwidth}
		\includegraphics[scale=0.42]{figures/data_for_thesis/DPD_double_band/errorA.pdf}
		\caption{}
		\label{fig:error_dpd_A}
	\end{subfigure}
	\hspace{2ex}
	\begin{subfigure}[b]{0.51\textwidth}
		\includegraphics[scale=0.42]{figures/data_for_thesis/DPD_double_band/errorB.pdf}
		\caption{}
		\label{fig:error_dpd_B}
	\end{subfigure}
	\caption{Спектральная плотность мощности сигнала нелинейных искажений: (\subref{fig:error_dpd_A}) канал A; (\subref{fig:error_dpd_B}) канал B}
	\label{fig:error_dpd_2ch}
\end{figure}

\subsection{Формирование обучающей выборки для задачи цифровых предыскажений в передатчике одноканальной системы связи в условиях динамически меняющегося режима работы усилителя мощности} \label{subsec:data_dpd_dynamic}

В реальных системах связи для соответствия требованиям протоколов связи~\cite{3gpp_36_804} производится динамическое распределение ресурсных блоков в рамках кадра данных в соответствии с реальным трафиком. Это распределение вызывает кратковременные изменения уровня мощности, что, в свою очередь, снижает коэффициент полезного действия (КПД) усилителя мощности. Указанный факт противоречит предположению о работе алгоритма цифрового предыскажения в стационарном режиме. Данная проблема исследуется в работе \cite{Dynamic_DPD_RB_Alloc}.

В то же время, для повышения энергоэффективности требуется увеличение динамического диапазона входной мощности УМ. Компенсация нелинейных искажений, индуцированных сигналами в широком динамическом диапазоне, требует повторной калибровки коэффициентов модели DPD в реальном времени, что на практике неэффективно с точки зрения распределения ресурсов системы.

Кроме того, переходные процессы, сопровождающие переключение режимов мощности УМ, генерируют нестационарные искажения. Эти искажения, передаваемые по каналу связи, являются причиной кратковременных нарушений протоколов связи и деградации спектральной маски излучаемого сигнала. Как следствие, прямое применение параметров модели DPD, полученных для одного режима мощности УМ, к сигналам другого режима приводит к значительной деградации характеристик системы.

Обучающая выборка для исследования динамического режима работы усилителя мощности состоит из комплесного OFDM сигнала шириной полосы 20~МГц со 100~ресурсными блоками на несущей частоте $f_{LO}~=~1.8 $~ГГц. Выборка включает в себя $C=61$ исходных сигналов передатчика в динамическом диапазоне 15 дБ:
\begin{align}
	&25.1 \text{ }мк\text{Вт}, \text{ }37.9\text{ }мк\text{Вт}, \text{ }50.8\text{ }мк\text{Вт}, \cdots, 794.3 \text{ }мк\text{Вт}, \nonumber \\
	&14.0 \text{ }\text{дБ}мк \text{ }15.8\text{ }\text{дБ}мк, \text{ }17.1\text{ }\text{дБ}мк, \cdots, 29.0 \text{ }\text{дБ}мк.
%	\label{dpd_dynamic_inp_powers}
\end{align}
Им соответствуют $C=61$ сигналов на выходе УМ в динамическом диапазоне~11.2~дБ:
\begin{align}
	&0.069 \text{ }\text{Вт}, \text{ }0.107\text{ }\text{Вт}, \text{ }0.143\text{ }\text{Вт}, \cdots, 0.912 \text{ }\text{Вт} \\
	-&11.6\text{ дБВт},-9.7\text{ дБВт},-8.4\text{ дБВт}, \cdots,-0.4\text{ дБВт} \nonumber
	\label{powers_all}
\end{align}
На рис.~\ref{fig:pa_power_dynamic_range} изображены распределения входной и выходной мощности УМ в линейном масштабе. Как видно из рис.~\ref{fig:pa_input_power_dynamic_range} мощность входного сигнала УМ распредлена равномерно. Для исключения смещения модели в процессе её обучения в пользу какого-либо уровня мощности, входная мощность УМ имеет равномерное распределение во всем динамическом диапазоне. Заметим, что согласно рис.~\ref{fig:pa_output_dynamic_range} распределение выходной мощности УМ отличается от равномерного ввиду нелинейности характеристики УМ.
\begin{figure} [h!]
	\begin{subfigure}[b]{0.51\textwidth}
		\includegraphics[scale=0.42]{figures/data_for_thesis/DPD_dynamic_single_band/input_power_dynamic_range.pdf}
		\caption{}
		\label{fig:pa_input_power_dynamic_range}
	\end{subfigure}
	\hspace{2ex}
	\begin{subfigure}[b]{0.51\textwidth}
		\includegraphics[scale=0.42]{figures/data_for_thesis/DPD_dynamic_single_band/output_power_dynamic_range.pdf}
		\caption{}
		\label{fig:pa_output_dynamic_range}
	\end{subfigure}
	\caption{Динамический диапазон мощности УМ: (\subref{fig:pa_input_power_dynamic_range}) входная мощность УМ; (\subref{fig:pa_output_dynamic_range}) выходная мощность УМ}
	\label{fig:pa_power_dynamic_range}
\end{figure}

На рис.~\ref{fig:dpd_dynamic_psd_data_example} изображены спектральные плотности мощности нормированного входного сигнала УМ~\ref{fig:dpd_dynamic_psd_tx} и нормированного выходного сигнала УМ~\ref{fig:dpd_dynamic_psd_pa_out} для 4-х случаев выходной мощности: -0.4 дБВт, -4.6 дБВт, -8.5 дБВт, -11.6 дБВт. Заметим, что 4-м случаям выходной мощности соответствуют неискаженные сигналы передатчика с мощностями 29.0 дБмк, 21.9 дБмк, 17.1 дБмк, 14.0 дБмк соответственно.  

Отметим, что графики СПМ входных сигналов УМ различной мощности совпадают с СПМ на рис.~\ref{fig:dpd_dynamic_psd_tx} с точностью до смещения в константу раз.

\begin{figure} [h!]
	\begin{subfigure}[b]{0.51\textwidth}
		\includegraphics[scale=0.42]{figures/data_for_thesis/DPD_dynamic_single_band/tx.pdf}
		\caption{}
		\label{fig:dpd_dynamic_psd_tx}
	\end{subfigure}
	\hspace{2ex}
	\begin{subfigure}[b]{0.51\textwidth}
		\includegraphics[scale=0.42]{figures/data_for_thesis/DPD_dynamic_single_band/pa_out.pdf}
		\caption{}
		\label{fig:dpd_dynamic_psd_pa_out}
	\end{subfigure}
	\caption{Спектральная плотность мощности: (\subref{fig:dpd_dynamic_psd_tx}) входного сигнала УМ; (\subref{fig:dpd_dynamic_psd_pa_out}) выходного сигнала УМ}
	\label{fig:dpd_dynamic_psd_data_example}
\end{figure}


%OFDM сигнал передатчика сформирован в соответствии со стандартом LTE. Согласно документации 3GPP \cite{3gpp_36_211} сигнал делится на ресурсные блоки по ширине полосы и длительности сигнала (англ. Resource Blocks).
%
%Каждый ресурсный блок содержит 12 поднесущих с сигналами с шириной полосы равной 15 кГц \cite{3gpp_36_211}. Таким образом, каждый такой блок имеет ширину полосы равную 180 кГц.
%
%В данной работе рассматриваются 5 различных сигналов передатчика. Каждый сигнал содержит $RB_{num}$ ресурсных блоков:
%\begin{equation*}
%	RB_{num}=\{1, 25, 50, 75, 100\}.
%\end{equation*}
%
%Кроме того, рассматриваются 3 различных импульсных характеристики, сгенерированные на языке программирования Python и соответствуют 3-м различным оценкам канала распространения паразитной помехи. Оценки канала были получены на тестовой платформе. Импульсные характеристики приведены в следующем разделе на рис. \ref{fig:path0}, \ref{fig:path1}, \ref{fig:path2}. Обозначим их соответственно:
%\begin{equation*}
%	multipath=\{path_0, path_1, path_2\}.
%\end{equation*}
%
%Таким образом, набор данных, на которых будет проводиться исследование алгоритмов компенсации помехи, содержит 15 различных сигналов паразитной помехи. В таблице \ref{tbl:signals} отображены условия, в которых формируются сигналы, имитирующие сигналы паразитной помехи. Эти сигналы, как отмечалось ранее, формируются в зависимости от ширины полосы сигнала передатчика и от канала распространения помехи от передатчика к приёмнику.
%\begin{table}[h]
%	\centering
%	\begin{tabular}{ | l | l | l | l | l | l |}
%		\hline
%		& \multicolumn{5}{ |c| }{$RB_{num}$}\\ \hline
%		& & & & & \\
%		\multirow{8}{*}{$multipath$} & 
%		RB1; $path_0$ & RB25; $path_0$ & RB50, $path_0$ & RB75; $path_0$ & RB100; $path_0$ \\
%		& & & & & \\ \cline{2-6}
%		& & & & & \\
%		& RB1; $path_1$ & RB25; $path_1$ & RB50; $path_1$ & RB75; $path_1$ & RB100; $path_1$ \\
%		& & & & & \\ \cline{2-6}
%		& & & & & \\
%		& RB1; $path_2$ & RB25; $path_2$ & RB50; $path_2$ & RB75; $path_2$ & RB100; $path_2$ \\
%		& & & & & \\ \hline		
%	\end{tabular}
%	\caption{Условия формирования сигналов, имитирующих сигнал паразитной помехи на приёмнике мобильного устройства}
%	\label{tbl:signals}
%\end{table}
%
%Отметим также, что ввиду требований стандарта LTE \cite{3gpp_36_211}, выходная мощность усилителя возрастает в зависимости от ширины полосы сигнала передатчика (таблица \ref{tbl:pwr_rb}).
%
%Исследование эффективности блочных и стохастических алгоритмов будем проводить в условиях стационарной паразитной помехи.
%
%\begin{table}[h]
%	\centering
%	\begin{tabular}{ | l | l | l | l | l | l |}
%		\hline
%		& & & & & \\
%		PA power, dBm & \ 6.7 & \ 11.7 & \ 17.2 & \ 22.8 & \ 26.8 \\ 
%		& & & & & \\ \hline
%		& & & & & \\
%		\ \ \ \ \ \ \ $RB_{num}$ & RB1 & RB25 & RB50 & RB75 & RB100 \\
%		& & & & & \\ \hline
%	\end{tabular}
%	\caption{Соответствие между выходной мощностью аналогового усилителя и количеством ресурсных блоков сигнала передатчика}
%	\label{tbl:pwr_rb}
%\end{table}
%
%Стационарность помехи означает, что условия формирования сигнала помехи (таблица \ref{tbl:signals}) должны быть неизменными на протяжении адаптации. То есть неизменными должны быть канал распространения помехи, а также ширина полосы сигнала передатчика.
%
%Запустим блочные и стохастические алгоритмы на тренировочных блоках данных, состоящих из 2457600 отсчётов сигналов, условия формирования которых отображены в таблице \ref{tbl:signals}. Затем сравним критерии адаптации, достигаемые методами первого порядка за это время с результатами, которые получены методами второго порядка.