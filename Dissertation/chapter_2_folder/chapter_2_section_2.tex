\section{Формирование тестового набора данных}
OFDM сигнал передатчика сформирован в соответствии со стандартом LTE. Согласно документации 3GPP \cite{3gpp_36_211} сигнал делится на ресурсные блоки по ширине полосы и длительности сигнала (англ. Resource Blocks).

Каждый ресурсный блок содержит 12 поднесущих с сигналами с шириной полосы равной 15 кГц \cite{3gpp_36_211}. Таким образом, каждый такой блок имеет ширину полосы равную 180 кГц.

В данной работе рассматриваются 5 различных сигналов передатчика. Каждый сигнал содержит $RB_{num}$ ресурсных блоков:
\begin{equation*}
	RB_{num}=\{1, 25, 50, 75, 100\}.
\end{equation*}

Кроме того, рассматриваются 3 различных импульсных характеристики, сгенерированные на языке программирования Python и соответствуют 3-м различным оценкам канала распространения паразитной помехи. Оценки канала были получены на тестовой платформе. Импульсные характеристики приведены в следующем разделе на рис. \ref{fig:path0}, \ref{fig:path1}, \ref{fig:path2}. Обозначим их соответственно:
\begin{equation*}
	multipath=\{path_0, path_1, path_2\}.
\end{equation*}

Таким образом, набор данных, на которых будет проводиться исследование алгоритмов компенсации помехи, содержит 15 различных сигналов паразитной помехи. В таблице \ref{tbl:signals} отображены условия, в которых формируются сигналы, имитирующие сигналы паразитной помехи. Эти сигналы, как отмечалось ранее, формируются в зависимости от ширины полосы сигнала передатчика и от канала распространения помехи от передатчика к приёмнику.
\begin{table}[h]
	\centering
	\begin{tabular}{ | l | l | l | l | l | l |}
		\hline
		& \multicolumn{5}{ |c| }{$RB_{num}$}\\ \hline
		& & & & & \\
		\multirow{8}{*}{$multipath$} & 
		RB1; $path_0$ & RB25; $path_0$ & RB50, $path_0$ & RB75; $path_0$ & RB100; $path_0$ \\
		& & & & & \\ \cline{2-6}
		& & & & & \\
		& RB1; $path_1$ & RB25; $path_1$ & RB50; $path_1$ & RB75; $path_1$ & RB100; $path_1$ \\
		& & & & & \\ \cline{2-6}
		& & & & & \\
		& RB1; $path_2$ & RB25; $path_2$ & RB50; $path_2$ & RB75; $path_2$ & RB100; $path_2$ \\
		& & & & & \\ \hline		
	\end{tabular}
	\caption{Условия формирования сигналов, имитирующих сигнал паразитной помехи на приёмнике мобильного устройства}
	\label{tbl:signals}
\end{table}

Отметим также, что ввиду требований стандарта LTE \cite{3gpp_36_211}, выходная мощность усилителя возрастает в зависимости от ширины полосы сигнала передатчика (таблица \ref{tbl:pwr_rb}).

Исследование эффективности блочных и стохастических алгоритмов будем проводить в условиях стационарной паразитной помехи.

\begin{table}[h]
	\centering
	\begin{tabular}{ | l | l | l | l | l | l |}
		\hline
		& & & & & \\
		PA power, dBm & \ 6.7 & \ 11.7 & \ 17.2 & \ 22.8 & \ 26.8 \\ 
		& & & & & \\ \hline
		& & & & & \\
		\ \ \ \ \ \ \ $RB_{num}$ & RB1 & RB25 & RB50 & RB75 & RB100 \\
		& & & & & \\ \hline
	\end{tabular}
	\caption{Соответствие между выходной мощностью аналогового усилителя и количеством ресурсных блоков сигнала передатчика}
	\label{tbl:pwr_rb}
\end{table}

Стационарность помехи означает, что условия формирования сигнала помехи (таблица \ref{tbl:signals}) должны быть неизменными на протяжении адаптации. То есть неизменными должны быть канал распространения помехи, а также ширина полосы сигнала передатчика.

Запустим блочные и стохастические алгоритмы на тренировочных блоках данных, состоящих из 2457600 отсчётов сигналов, условия формирования которых отображены в таблице \ref{tbl:signals}. Затем сравним критерии адаптации, достигаемые методами первого порядка за это время с результатами, которые получены методами второго порядка.