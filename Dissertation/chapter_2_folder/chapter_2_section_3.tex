\section{Исследование алгоритмов и адаптивных моделей компенсации паразитной помехи на приёмнике  устройств связи}

\subsection{Компенсация паразитной помехи на приёмнике полнодуплексной системы связи методом адаптации модели Гаммерштейна}

В данном разделе рассмативается задача компенсации паразитных помех, возникающих в приёмнике полнодуплексной системы связи. Процесс возникновения помехи, а также тестовая установка формирования паразитных помех данного типа описаны в разделах~\ref{sec:testbench_descr},~\ref{subsec:data_ibfd} главы~\ref{chapter:methods}.

Адаптивная модель~\cite{behav_model} компенсации паразитной помехи строится таким образом, чтобы отражать физические процессы возникновения данного рода помех. Согласно схемы возникновения и компенсации паразитных помех на приёмнике на рис.~\ref{fig:ident_problem} в сигнал передатчика искажается вследствие нелинейности характеристик усилителя передатчика и дуплексера. Затем сигнал искажается вследствие распространения по каналу TX-RX.

Таким образом, модель Винера-Гаммерштейна ~\cite{behav_model}, представленная на рис.~\ref{fig:wiener_hammerstein} отражает процесс возникновения собственной помехи в примёном тракте и может быть использована для решения задачи идентификации помехи. 

Согласно результатам эксперимента, представленным в текущем разделе, эффекты инерционности УМ проявляются слабо. В связи с этим в данном эксперименте использована модель Гаммерштейна~(рис.~\ref{fig:hammerstein}), которая является упрощением модели Винера-Гаммерштейна~(рис.~\ref{fig:wiener_hammerstein}) для случая нелинейных эффектов без памяти. 

В качестве нелинейного блока модели Гаммерштейна~\eqref{hammerstein_output} выбраны линейные сплайновые полиномы, описанные в разделе~\ref{subsec_splines} главы~\ref{chapter:intro}. Линейно-сплайновые полиномы в данном эксперименте выбраны ввиду его низкой вычислительной сложности и простоты аппаратной реализации~\cite{dpd_models, lut_dpd}.

Для экспериментов была выбрана модель Гаммерштейна с адаптивным фильтром c $L=45$ числом параметров, а также кусочно-линейной функцией, содержащей $P=8$ сплайнов первого порядка. 

Полный набор данных состоит из $80000$ отсчётов, 50\% из которых используется как тренировочная выборка, остальные отсчёты используются как тестовая выборка.

В данном эксперименте проводится сравнение блочного градиентного спуска (BGD, Block Gradient Descent), стохастического градиентного спуска (SGD, Stochastic Gradient Descent), описанных в разделе~\ref{subsec:grad} главы~\ref{chapter:intro} со смешанным методом Ньютона~(раздел~\ref{subsec:mnm} главы~\ref{chapter:intro}). Отметим, что ввиду голоморфности выхода модели Гаммерштейна~\eqref{hammerstein_output}, метод градиентного спуска для адаптации данной модели по методу наименьших квадратов представлен выражением~\eqref{grad_descent_сomplex_mse_holomorphic}. Отметим также, что в данном эксперименте отличие блочного и стоъастического градиентного спуска заключается в формировании блока данных для шага адаптации. 

Шаг адаптации смешанного метода Ньютона и BGD пересчитывается на всей тренировочной последовательности. Другими словами, якобиан выхода модели по параметрам модели $D_{\bit{z}}\bit{y}$, представленный выражением~\eqref{hammerst_jacobian_full}, а также вектор ошибки~\eqref{grad_descent_сomplex_mse_holomorphic} вычисляются для $N=40000$ отсчётов.

С другой стороны, в случае стохастического градиентного спуска вся тренировочная последовательность делится на неперекрывающиеся блоки, которые упаковываются в 50~блоков по 200~отсчётов каждый. Таким образом, за одну эпоху (одно прохождение всей тренировочной последовательности) модель тренируется на 4~блоках. Градиенты вычисляются для каждого блока (по 200 отсчётов) и усредняются (по 50 блоков). Размер блока был выбран экспериментально с точки зрения максимизации результирующего уровня качества модели.

Также следует обратить внимание, что в случае блочного градиентного спуска и смешанного метода Ньютона параметры обновляются один раз за эпоху, то есть после подсчёта шага на полной последовательности тренировочных данных. В то время как при реализации стохастического градиентного спуска параметры пересчитываются после вычисления стохастического градиента на каждом блоке ($50\times200$ отсчётов).

Кроме того, в данном численном эксперименте рассмотрены ускоренные версии градиентного спуска, представленные оптимизаторами Adam и Momentum, описанными в разделе~\ref{subsec:grad_modified} главы~\ref{chapter:intro}.

В следующих симуляциях нормированная среднеквадратичная ошибка NMSE~\eqref{nmse_block} отслеживается на каждой эпохе на тренировочной и тестовой последовательности, что отражено на рис.~\ref{lc_mnm_bgd},~\ref{lc_mnm_sgd}.
\begin{figure}[h!]
	\centering
	\captionsetup{justification=centering}
	\centerline{\includegraphics[width = 0.65\textwidth]{figures/results/mnm_crm/mnm_bgd.pdf}}
	\caption{Кривые сходимости на тренировочной и тестовой последовательностях. Сравнение BGD с оптимизаторами Momentum, Adam и смешанного метода Ньютона. BGD -- блочный градиентный спуск, MNM -- смешанный метод Ньютона, NMSE -- нормированная среднеквадратичная ошибка, Adam, Momentum -- оптимизаторы градиентного спуска.}
	\label{lc_mnm_bgd}
\end{figure}
\begin{figure}[h!]
	\centering
	\captionsetup{justification=centering}
	\centerline{\includegraphics[width = 0.65\textwidth]{figures/results/mnm_crm/mnm_sgd.pdf}}
	\caption{Кривые сходимости на тренировочной и тестовой последовательностях. Сравнение SGD с оптимизаторами Momentum, Adam и смешанного метода Ньютона. SGD -- стохастический градиентный спуск, MNM -- смешанный метод Ньютона, NMSE -- нормированная среднеквадратичная ошибка, Adam, Momentum -- оптимизаторы градиентного спуска.}
	\label{lc_mnm_sgd}
\end{figure}

Сравнение скорости сходимости блочного градиентного спуска с оптимизаторами Momentum и Adam, а также смешанного метода Ньютона показано на рис.~\ref{lc_mnm_bgd}. Смешанный метод Ньютона требует 30 эпох для достижения результирующего значения уровня компенсации, в то время как блочным градиентный спуск с оптимизаторами требуют $\sim10000$ эпох для достижения такого же значения NMSE.

На рис.~\ref{lc_mnm_sgd} представлены кривые обучения, полученные для стохастического градиентного спуска с оптимизаторами Momentum и Adam и смешанного метода Ньютоном. Графики сходимости отражают тот факт, что SGD также требует приблизительно $10000$ эпох для достижения уровня компенсации паразитной помехи, полученного при помощи смешанного метода Ньютона.

Отметим, что параметры оптимизаторов и темп обучения градиентного спуска, были выбраны с точки зрения наилучшего результирующего качества модели и высокой скорости сходимости.

Спектральные плотности мощности (СПМ) исходной и компенсированной различными методами помехи изображены на рис.~\ref{psd}. Графики СПМ построены на основе тестовой последовательности.
\begin{figure}[h!]
	\centering
	\captionsetup{justification=centering}
	\centerline{\includegraphics[width = 0.65\textwidth]{figures/results/mnm_crm/psd.pdf}}
	\caption{Спектральные плотности мощности исходной и компенсированных помех. BGD -- блочный градиентный спуск, SGD -- стохастический градиентный спуск, Adam, Momentum -- оптимизаторы градиентного спуска, MNM -- смешанный метод Ньютона, NMSE -- нормированная среднеквадратичная ошибка.}
	\label{psd}
\end{figure}

Таблица~\ref{table_of_results} отражает заметное увеличение скорости сходимости смешанного метода Ньютона по сравнению с классическими градиентными методами. Для достижения сопоставимого уровня компенсации для смешанного метода Ньютона требуется всего 30 эпох. Несмотря на то, что при использовании метода второго порядка требуется существенно меньшее число шагов оптимизации, каждый шаг адаптации вычисляется~$\sim5$ в раз дольше. Тем не менее, общее время, затрачиваемое на обучение модели, существенно меньше по сравнению с методами первого порядка.
\begin{table}[h!]
	\centering
	\caption{Сравнение уровня компенсации помехи и скорости сходимости}
	\begin{tabular}{|l|c|c|c|c|c|}
		\hline
		\textbf{Алгоритм} &
		\textbf{\begin{tabular}[c]{@{}c@{}}BGD\\ Moment.\end{tabular}} &
		\textbf{\begin{tabular}[c]{@{}c@{}}BGD\\ Adam\end{tabular}} &
		\textbf{\begin{tabular}[c]{@{}c@{}}SGD \\ Moment.\end{tabular}} &
		\textbf{\begin{tabular}[c]{@{}c@{}}SGD \\ Adam\end{tabular}} &
		\textbf{MNM} \\ \hline
		\textbf{\begin{tabular}[c]{@{}l@{}}Номер\\ эпохи\end{tabular}} &
		{\color[HTML]{000000} 10000} &
		{\color[HTML]{000000} 10000} &
		{\color[HTML]{000000} 10000} &
		10000 &
		30 \\ \hline
		\textbf{\begin{tabular}[c]{@{}l@{}}Время за\\ эпоху, $10^{-2}$ с\end{tabular}} &
		{\color[HTML]{000000} 3.8} &
		{\color[HTML]{000000} 4.0} &
		{\color[HTML]{000000} 3.7} &
		4.1 &
		21 \\ \hline
		\textbf{\begin{tabular}[c]{@{}l@{}}Общее время\\ работы, с\end{tabular}} &
		{\color[HTML]{000000} 380} &
		{\color[HTML]{000000} 403} &
		{\color[HTML]{000000} 386} &
		412 &
		6.2 \\ \hline
		\multirow{2}{*}{\textbf{NMSE, dB}} &
		\multirow{2}{*}{-45.1} &
		\multirow{2}{*}{-46.7} &
		\multirow{2}{*}{-46.6} &
		\multirow{2}{*}{-46.5} &
		\multirow{2}{*}{-46.9} \\ 
		& & & & & \\ \hline 
	\end{tabular}
	\label{table_of_results}
\end{table}

Как было отмечено ранее, итерация смешанного метода Ньютона подразумевает обращение матрицы Гёссе~\eqref{mixed_newton_eq_jac}. В связи с этим вычислительная сложность шага смешанного метода Ньютона~\cite{mixed_newton_global_opt} высока по сравнению с алгоритмами на основе градиентного спуска~\cite{polyak1964some},~\cite{kingma2014adam}. Тем не менее, число шагов, необходимых для достижения такого же уровня компенсации, значительно меньше для смешанного Ньютона (таблица~\ref{table_of_results}). Таким образом, смешанный метод Ньютона отлично подходит для исследования возможного уровня качества моделей. Кроме того, смешанный метод Ньютона можно использовать для онлайн обучения моделей, так как в реальных приложениях требуется накопление матрицы Гёссе и, как следствие, частота шагов оптимизации существенно меньше по сравнению с методами первого порядка.

Таким образом, несмотря на то, что вычисление шага оптимизации смешанного метода Ньютона требует в~$\sim5$~раз больше времени, общая длительность обучения уменьшена до 30 эпох, по сравнению 10000 эпох необходимых для классических методов оптимизации первого порядка, для достижения того же уровня компенсации $46.9$ дБ.

\subsection{Компенсация паразитной помехи 2-ого порядка на основе адаптации классических и нейросетевых структур}



Сравнение уровня компенсации паразитной помехи различными алгоритмами будем проводить относительно критерия нормированного среднего квадрата ошибки \eqref{nmse_block}, поскольку NMSE позволяет оценить уровень отклонения выхода модели от помехи на приёмнике независимо от динамического диапазона помехи.

Методы второго порядка обеспечивают быстрое приближение к точке оптимума по сравнению с методами первого порядка. В связи с этим целесообразно использовать значения NMSE, полученные в результате работы методов второго порядка, в качестве опорных уровней подавления помехи для дальнейшего сравнения с ними значений NMSE, полученных градиентными методами первого порядка. 

Следует отметить, что метод Ньютона может стогнировать при определенных условиях. Это может произойти, например, в случае если коэффициенты модели примут значения, соответствующие точке в окрестности седловой точки целевой функции. В этом случае глобальный минимум не может быть достигнут в результате работы метода Ньютона.

На рис. \ref{fig:adapt_curves_rb25_path0_stogn} изображены кривые адаптации метода Ньютона и демпфированного метода Ньютона в случае, когда сигнал передатчика имеет ширину полосы равную 4.5 МГц (RB25), а канал распространения помехи $path_0$. При этом сигналы передатчика и помехи состоят из 20 блоков (рис. \ref{fig:adapt_curves_rb25_path0_stogn_20_epoch}) и 400 блоков (рис. \ref{fig:adapt_curves_rb25_path0_stogn_400_epoch}) длиной 122880 отсчётов.
\begin{figure}
	\begin{subfigure}[b]{0.51\textwidth}
		\includegraphics[scale=0.5]{figures/adapt_curves/newton_simple_stogn_20_epoch.pdf}
		\caption{Сигналы помехи и на передатчике состоят из 20 блоков длиной 122880 отсчётов}
		\label{fig:adapt_curves_rb25_path0_stogn_20_epoch}
	\end{subfigure}
	\hspace{2ex}
	\begin{subfigure}[b]{0.51\textwidth}
		\includegraphics[scale=0.5]{figures/adapt_curves/newton_simple_stogn_400_epoch.pdf}
		\caption{Сигналы помехи и на передатчике состоят из 400 блоков длиной 122880 отсчётов}
		\label{fig:adapt_curves_rb25_path0_stogn_400_epoch}
	\end{subfigure}
	\caption{Кривые адаптации модели Гаммерштейна методом Ньютона и демпфированным методов Ньютона. Случай $\{\text{RB25}, path_0\}$}
	\label{fig:adapt_curves_rb25_path0_stogn}
\end{figure}

Таким образом, предпочтительней использовать значения критерия NMSE, полученные в результате работы демпфированного метода Ньютона в качестве опорных значений.

В дальнейшем при исследовании работы стохастических алгоритмов будем сравнивать значения критерия NMSE, получаемые в результате работы алгоритмов SGD-SGD, SGD-DCD со значениями NMSE, получаемыми в результате работы демпфированного метода Ньютона.

Вычислим опорные значения критерия NMSE, полученные при помощи демпфированного метода Ньютона для каждого случая, отображённого в таблице \ref{tbl:signals}.

Рассмотрим подавление паразитной помехи в случае~$\{\text{RB1},~path_0\}$~(рис.~\ref{fig:cond_rb1_path0}), когда ширина полосы сигнала передатчика определяется одним ресурсным блоком, 180~кГц~(рис. \ref{fig:tx_rb1}), а канал распространения определяется частотной характеристикой $path0$~(рис. \ref{fig:path0}).
\begin{figure}[h!]
	\begin{subfigure}[h!]{0.51\textwidth}
		\includegraphics[scale=0.45]{figures/tx/tx_rb1.pdf}
		\caption{Спектральная плотность мощности сигнала передатчика с одним ресурсным блоком}
		\label{fig:tx_rb1}
	\end{subfigure}
	\hspace{2ex}
	\begin{subfigure}[h!]{0.51\textwidth}
		\includegraphics[scale=0.45]{figures/freq_resp/path0.pdf}
		\caption{Амплитудно-частотная характеристика канала распространения помехи $path_0$}
		\label{fig:path0}
	\end{subfigure}
	\caption{Условия формирования сигнала помехи на приёмнике $\{\text{RB1}, path_0\}$}
	\label{fig:cond_rb1_path0}
\end{figure}

На рис. \ref{fig:psd_rb1_path0_newton} изображены спектральные плотности мощности паразитной помехи, сигнала на выходе модели Гаммерштейна после адаптации демпфированным методом Ньютона, а также отклонения выхода модели Гаммерштейна от сигнала помехи. Значение критерия NMSE в данном случае составляет $-23.8$ dB.
\begin{figure}[h!]
	\begin{subfigure}[h!]{0.51\textwidth}
		\includegraphics[scale=0.45]{figures/psd/psd_newton_rb1_path0.pdf}
		\caption{}
		\label{fig:psd_rb1_path0_newton_reduced}
	\end{subfigure}
	\hspace{2ex}
	\begin{subfigure}[h!]{0.51\textwidth}
		\includegraphics[scale=0.45]{figures/psd/psd_newton_rb1_path0_expanded.pdf}
		\caption{}
		\label{fig:psd_rb1_path0_newton_expanded}
	\end{subfigure}
	\caption{Спектральная плотность мощности сигнала паразитной помехи после адаптации методом Ньютона. Случай $\{\text{RB50}, path_1\}$}
	\label{fig:psd_rb1_path0_newton}
\end{figure}

Рассмотрим подавление паразитной помехи в случае~$\{\text{RB50},~path_1\}$~(рис.~\ref{fig:cond_rb50_path1}), когда ширина полосы сигнала передатчика определяется 50-ю ресурсными блоками, 9~МГц~(рис.~\ref{fig:tx_rb50}), а канал распространения определяется частотной характеристикой $path1$ (рис. \ref{fig:path1}).
\begin{figure}
	\begin{subfigure}[b]{0.51\textwidth}
		\includegraphics[scale=0.45]{figures/tx/tx_rb50.pdf}
		\caption{Спектральная плотность мощности сигнала передатчика с 50-ю ресурсными блоками}
		\label{fig:tx_rb50}
	\end{subfigure}
	\hspace{2ex}
	\begin{subfigure}[b]{0.51\textwidth}
		\includegraphics[scale=0.45]{figures/freq_resp/path1.pdf}
		\caption{Амплитудно-частотная характеристика канала распространения помехи $path_1$}
		\label{fig:path1}
	\end{subfigure}
	\caption{Условия формирования сигнала помехи на приёмнике $\{\text{RB50}, path_1\}$}
	\label{fig:cond_rb50_path1}
\end{figure}

На рис. \ref{fig:psd_rb50_path1_newton} изображены спектральные плотности мощности паразитной помехи, сигнала на выходе модели Гаммерштейна после адаптации демпфированным методом Ньютона, а также отклонения выхода модели Гаммерштейна от сигнала помехи. Значение критерия NMSE в данном случае составляет $-36.2$ dB.
\begin{figure}
	\begin{subfigure}[b]{0.51\textwidth}
		\includegraphics[scale=0.45]{figures/psd/psd_newton_rb50_path1.pdf}
		\caption{}
		\label{fig:psd_rb50_path1_newton_reduced}
	\end{subfigure}
	\hspace{2ex}
	\begin{subfigure}[b]{0.51\textwidth}
		\includegraphics[scale=0.45]{figures/psd/psd_newton_rb50_path1_expanded.pdf}
		\caption{}
		\label{fig:psd_rb50_path1_newton_expanded}
	\end{subfigure}
	\caption{Спектральная плотность мощности сигнала паразитной помехи после адаптации методом Ньютона. Случай $\{\text{RB50}, path_1\}$}
	\label{fig:psd_rb50_path1_newton}
\end{figure}

Рассмотрим подавление паразитной помехи в случае $\{\text{RB100}, path_2\}$ (рис. \ref{fig:cond_rb100_path2}), когда ширина полосы сигнала передатчика определяется 100 ресурсными блоками, 18~МГц~(рис.~\ref{fig:tx_rb100}), а канал распространения определяется частотной характеристикой $path2$~(рис.~\ref{fig:path2}).
\begin{figure}
	\begin{subfigure}[b]{0.51\textwidth}
		\includegraphics[scale=0.45]{figures/tx/tx_rb100.pdf}
		\caption{Спектральная плотность мощности сигнала передатчика с 100 ресурсными блоками}
		\label{fig:tx_rb100}
	\end{subfigure}
	\hspace{2ex}
	\begin{subfigure}[b]{0.51\textwidth}
		\includegraphics[scale=0.45]{figures/freq_resp/path2.pdf}
		\caption{Амплитудно-частотная характеристика канала распространения помехи $path_2$}
		\label{fig:path2}
	\end{subfigure}
	\caption{Условия формирования сигнала помехи на приёмнике $\{\text{RB100}, path_2\}$}
	\label{fig:cond_rb100_path2}
\end{figure}

На рис. \ref{fig:psd_rb100_path2_newton} изображены спектральные плотности мощности паразитной помехи, сигнала на выходе модели Гаммерштейна после адаптации демпфированным методом Ньютона, а также отклонения выхода модели Гаммерштейна от сигнала помехи. Значение критерия NMSE в данном случае составляет $-34.3$ dB.
\begin{figure}
	\begin{subfigure}[b]{0.51\textwidth}
		\includegraphics[scale=0.45]{figures/psd/psd_newton_rb100_path2.pdf}
		\caption{}
		\label{fig:psd_rb100_path2_newton_reduced}
	\end{subfigure}
	\hspace{2ex}
	\begin{subfigure}[b]{0.51\textwidth}
		\includegraphics[scale=0.39]{figures/psd/psd_newton_rb100_path2_expanded.pdf}
		\caption{}
		\label{fig:psd_rb100_path2_newton_expanded}
	\end{subfigure}
	\caption{Спектральная плотность мощности сигнала паразитной помехи после адаптации методом Ньютона. Случай $\{\text{RB100}, path_2\}$}
	\label{fig:psd_rb100_path2_newton}
\end{figure}

Спектральные плотности мощности помехи до и после подавления паразитной помехи демпфированным методом Ньютона для других случаев, приведённых в таблице \ref{tbl:signals}, представлены в приложении А1.

Таблица \ref{tbl:reference_nmse} отражает опорные значения критерия NMSE dB, получаемые в результате адаптации модели Гаммерштейна демпфированным методом Ньютона для всех случаев формирования паразитной помехи, представленных в таблице~\ref{tbl:signals}.
\begin{table}[h]
	\centering
	\begin{tabular}{ | l | l | l | l | l | l |}
		\hline
		& RB1 & RB25 & RB50 & RB75 & RB100 \\ \hline
		& & & & & \\
		$path_0$ & -24.1 & -33.2 & -31.7 & -31.8 & -29.7 \\
		& & & & & \\ \hline
		& & & & & \\
		$path_1$ & -22.2 & -31.7 & -31.6 & -31.3 & -29.2 \\
		& & & & & \\ \hline
		& & & & & \\
		$path_2$ & -25.4 & -34.1 & -31.8 & -32.0 & -30.0 \\
		& & & & & \\ \hline
	\end{tabular}
	\caption{Опорные значения критерия NMSE dB, полученные в результате адаптации демпфированным методом Ньютона для каждого случая формирования паразитной помехи}
	\label{tbl:reference_nmse}
\end{table} %%%%%%%%%%%%%

Таким образом, в результате работы демпфированного метода Ньютона подавление паразитной помехи для сигнала шириной полосы 180 кГц не хуже 22 дБ для трех исследованных характеристик линейного канала распространения. 

Подавление помехи в случае сигналов передатчика с шириной полосы 4.5 МГц, 9 МГц и 13.5 МГц не хуже 31 дБ для трёх рассмотренных оценок канала распространения помехи.

В случае сигнала передатчика с шириной полосы 18 МГц подавление помехи демпфированным методом Ньютона не хуже 29 дБ также для всех рассмотренных оценок канала распространения. 