\section{Исследование алгоритмов компенсации стационарной паразитной помехи на приёмнике}

\subsection{Аппроксимация стационарных нелинейных искажений на приемнике устройства связи на основе классических и нейросетевых структур}

\subsection{Исследование опорных уровней компенсации стационарной помехи методами второго порядка}
Сравнение уровня компенсации паразитной помехи различными алгоритмами будем проводить относительно критерия нормированного среднего квадрата ошибки \eqref{nmse_block}, поскольку NMSE позволяет оценить уровень отклонения выхода модели от помехи на приёмнике независимо от динамического диапазона помехи.

Методы второго порядка обеспечивают быстрое приближение к точке оптимума по сравнению с методами первого порядка. В связи с этим целесообразно использовать значения NMSE, полученные в результате работы методов второго порядка, в качестве опорных уровней подавления помехи для дальнейшего сравнения с ними значений NMSE, полученных градиентными методами первого порядка. 

Следует отметить, что метод Ньютона может стогнировать при определенных условиях. Это может произойти, например, в случае если коэффициенты модели примут значения, соответствующие точке в окрестности седловой точки целевой функции. В этом случае глобальный минимум не может быть достигнут в результате работы метода Ньютона.

На рис. \ref{fig:adapt_curves_rb25_path0_stogn} изображены кривые адаптации метода Ньютона и демпфированного метода Ньютона в случае, когда сигнал передатчика имеет ширину полосы равную 4.5 МГц (RB25), а канал распространения помехи $path_0$. При этом сигналы передатчика и помехи состоят из 20 блоков (рис. \ref{fig:adapt_curves_rb25_path0_stogn_20_epoch}) и 400 блоков (рис. \ref{fig:adapt_curves_rb25_path0_stogn_400_epoch}) длиной 122880 отсчётов.
\begin{figure}
	\begin{subfigure}[b]{0.51\textwidth}
		\includegraphics[scale=0.5]{figures/adapt_curves/newton_simple_stogn_20_epoch.pdf}
		\caption{Сигналы помехи и на передатчике состоят из 20 блоков длиной 122880 отсчётов}
		\label{fig:adapt_curves_rb25_path0_stogn_20_epoch}
	\end{subfigure}
	\hspace{2ex}
	\begin{subfigure}[b]{0.51\textwidth}
		\includegraphics[scale=0.5]{figures/adapt_curves/newton_simple_stogn_400_epoch.pdf}
		\caption{Сигналы помехи и на передатчике состоят из 400 блоков длиной 122880 отсчётов}
		\label{fig:adapt_curves_rb25_path0_stogn_400_epoch}
	\end{subfigure}
	\caption{Кривые адаптации модели Гаммерштейна методом Ньютона и демпфированным методов Ньютона. Случай $\{\text{RB25}, path_0\}$}
	\label{fig:adapt_curves_rb25_path0_stogn}
\end{figure}

Таким образом, предпочтительней использовать значения критерия NMSE, полученные в результате работы демпфированного метода Ньютона в качестве опорных значений.

В дальнейшем при исследовании работы стохастических алгоритмов будем сравнивать значения критерия NMSE, получаемые в результате работы алгоритмов SGD-SGD, SGD-DCD со значениями NMSE, получаемыми в результате работы демпфированного метода Ньютона.

Вычислим опорные значения критерия NMSE, полученные при помощи демпфированного метода Ньютона для каждого случая, отображённого в таблице \ref{tbl:signals}.

Рассмотрим подавление паразитной помехи в случае~$\{\text{RB1},~path_0\}$~(рис.~\ref{fig:cond_rb1_path0}), когда ширина полосы сигнала передатчика определяется одним ресурсным блоком, 180~кГц~(рис. \ref{fig:tx_rb1}), а канал распространения определяется частотной характеристикой $path0$~(рис. \ref{fig:path0}).
\begin{figure}[h!]
	\begin{subfigure}[h!]{0.51\textwidth}
		\includegraphics[scale=0.45]{figures/tx/tx_rb1.pdf}
		\caption{Спектральная плотность мощности сигнала передатчика с одним ресурсным блоком}
		\label{fig:tx_rb1}
	\end{subfigure}
	\hspace{2ex}
	\begin{subfigure}[h!]{0.51\textwidth}
		\includegraphics[scale=0.45]{figures/freq_resp/path0.pdf}
		\caption{Амплитудно-частотная характеристика канала распространения помехи $path_0$}
		\label{fig:path0}
	\end{subfigure}
	\caption{Условия формирования сигнала помехи на приёмнике $\{\text{RB1}, path_0\}$}
	\label{fig:cond_rb1_path0}
\end{figure}

На рис. \ref{fig:psd_rb1_path0_newton} изображены спектральные плотности мощности паразитной помехи, сигнала на выходе модели Гаммерштейна после адаптации демпфированным методом Ньютона, а также отклонения выхода модели Гаммерштейна от сигнала помехи. Значение критерия NMSE в данном случае составляет $-23.8$ dB.
\begin{figure}[h!]
	\begin{subfigure}[h!]{0.51\textwidth}
		\includegraphics[scale=0.45]{figures/psd/psd_newton_rb1_path0.pdf}
		\caption{}
		\label{fig:psd_rb1_path0_newton_reduced}
	\end{subfigure}
	\hspace{2ex}
	\begin{subfigure}[h!]{0.51\textwidth}
		\includegraphics[scale=0.45]{figures/psd/psd_newton_rb1_path0_expanded.pdf}
		\caption{}
		\label{fig:psd_rb1_path0_newton_expanded}
	\end{subfigure}
	\caption{Спектральная плотность мощности сигнала паразитной помехи после адаптации методом Ньютона. Случай $\{\text{RB50}, path_1\}$}
	\label{fig:psd_rb1_path0_newton}
\end{figure}

Рассмотрим подавление паразитной помехи в случае~$\{\text{RB50},~path_1\}$~(рис.~\ref{fig:cond_rb50_path1}), когда ширина полосы сигнала передатчика определяется 50-ю ресурсными блоками, 9~МГц~(рис.~\ref{fig:tx_rb50}), а канал распространения определяется частотной характеристикой $path1$ (рис. \ref{fig:path1}).
\begin{figure}
	\begin{subfigure}[b]{0.51\textwidth}
		\includegraphics[scale=0.45]{figures/tx/tx_rb50.pdf}
		\caption{Спектральная плотность мощности сигнала передатчика с 50-ю ресурсными блоками}
		\label{fig:tx_rb50}
	\end{subfigure}
	\hspace{2ex}
	\begin{subfigure}[b]{0.51\textwidth}
		\includegraphics[scale=0.45]{figures/freq_resp/path1.pdf}
		\caption{Амплитудно-частотная характеристика канала распространения помехи $path_1$}
		\label{fig:path1}
	\end{subfigure}
	\caption{Условия формирования сигнала помехи на приёмнике $\{\text{RB50}, path_1\}$}
	\label{fig:cond_rb50_path1}
\end{figure}

На рис. \ref{fig:psd_rb50_path1_newton} изображены спектральные плотности мощности паразитной помехи, сигнала на выходе модели Гаммерштейна после адаптации демпфированным методом Ньютона, а также отклонения выхода модели Гаммерштейна от сигнала помехи. Значение критерия NMSE в данном случае составляет $-36.2$ dB.
\begin{figure}
	\begin{subfigure}[b]{0.51\textwidth}
		\includegraphics[scale=0.45]{figures/psd/psd_newton_rb50_path1.pdf}
		\caption{}
		\label{fig:psd_rb50_path1_newton_reduced}
	\end{subfigure}
	\hspace{2ex}
	\begin{subfigure}[b]{0.51\textwidth}
		\includegraphics[scale=0.45]{figures/psd/psd_newton_rb50_path1_expanded.pdf}
		\caption{}
		\label{fig:psd_rb50_path1_newton_expanded}
	\end{subfigure}
	\caption{Спектральная плотность мощности сигнала паразитной помехи после адаптации методом Ньютона. Случай $\{\text{RB50}, path_1\}$}
	\label{fig:psd_rb50_path1_newton}
\end{figure}

Рассмотрим подавление паразитной помехи в случае $\{\text{RB100}, path_2\}$ (рис. \ref{fig:cond_rb100_path2}), когда ширина полосы сигнала передатчика определяется 100 ресурсными блоками, 18~МГц~(рис.~\ref{fig:tx_rb100}), а канал распространения определяется частотной характеристикой $path2$~(рис.~\ref{fig:path2}).
\begin{figure}
	\begin{subfigure}[b]{0.51\textwidth}
		\includegraphics[scale=0.45]{figures/tx/tx_rb100.pdf}
		\caption{Спектральная плотность мощности сигнала передатчика с 100 ресурсными блоками}
		\label{fig:tx_rb100}
	\end{subfigure}
	\hspace{2ex}
	\begin{subfigure}[b]{0.51\textwidth}
		\includegraphics[scale=0.45]{figures/freq_resp/path2.pdf}
		\caption{Амплитудно-частотная характеристика канала распространения помехи $path_2$}
		\label{fig:path2}
	\end{subfigure}
	\caption{Условия формирования сигнала помехи на приёмнике $\{\text{RB100}, path_2\}$}
	\label{fig:cond_rb100_path2}
\end{figure}

На рис. \ref{fig:psd_rb100_path2_newton} изображены спектральные плотности мощности паразитной помехи, сигнала на выходе модели Гаммерштейна после адаптации демпфированным методом Ньютона, а также отклонения выхода модели Гаммерштейна от сигнала помехи. Значение критерия NMSE в данном случае составляет $-34.3$ dB.
\begin{figure}
	\begin{subfigure}[b]{0.51\textwidth}
		\includegraphics[scale=0.45]{figures/psd/psd_newton_rb100_path2.pdf}
		\caption{}
		\label{fig:psd_rb100_path2_newton_reduced}
	\end{subfigure}
	\hspace{2ex}
	\begin{subfigure}[b]{0.51\textwidth}
		\includegraphics[scale=0.39]{figures/psd/psd_newton_rb100_path2_expanded.pdf}
		\caption{}
		\label{fig:psd_rb100_path2_newton_expanded}
	\end{subfigure}
	\caption{Спектральная плотность мощности сигнала паразитной помехи после адаптации методом Ньютона. Случай $\{\text{RB100}, path_2\}$}
	\label{fig:psd_rb100_path2_newton}
\end{figure}

Спектральные плотности мощности помехи до и после подавления паразитной помехи демпфированным методом Ньютона для других случаев, приведённых в таблице \ref{tbl:signals}, представлены в приложении А1.

Таблица \ref{tbl:reference_nmse} отражает опорные значения критерия NMSE dB, получаемые в результате адаптации модели Гаммерштейна демпфированным методом Ньютона для всех случаев формирования паразитной помехи, представленных в таблице~\ref{tbl:signals}.
\begin{table}[h]
	\centering
	\begin{tabular}{ | l | l | l | l | l | l |}
		\hline
		& RB1 & RB25 & RB50 & RB75 & RB100 \\ \hline
		& & & & & \\
		$path_0$ & -24.1 & -33.2 & -31.7 & -31.8 & -29.7 \\
		& & & & & \\ \hline
		& & & & & \\
		$path_1$ & -22.2 & -31.7 & -31.6 & -31.3 & -29.2 \\
		& & & & & \\ \hline
		& & & & & \\
		$path_2$ & -25.4 & -34.1 & -31.8 & -32.0 & -30.0 \\
		& & & & & \\ \hline
	\end{tabular}
	\caption{Опорные значения критерия NMSE dB, полученные в результате адаптации демпфированным методом Ньютона для каждого случая формирования паразитной помехи}
	\label{tbl:reference_nmse}
\end{table} %%%%%%%%%%%%%

Таким образом, в результате работы демпфированного метода Ньютона подавление паразитной помехи для сигнала шириной полосы 180 кГц не хуже 22 дБ для трех исследованных характеристик линейного канала распространения. 

Подавление помехи в случае сигналов передатчика с шириной полосы 4.5 МГц, 9 МГц и 13.5 МГц не хуже 31 дБ для трёх рассмотренных оценок канала распространения помехи.

В случае сигнала передатчика с шириной полосы 18 МГц подавление помехи демпфированным методом Ньютона не хуже 29 дБ также для всех рассмотренных оценок канала распространения. 