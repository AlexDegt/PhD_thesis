\section{Структура тестовой платформы}

В данном разделе описывается тестовая установка, на которой сформирован сигнал, имитирующий сигнал паразитной помехи в приёмном тракте мобильного устройства. 

На рис. \ref{fig:install} изображена установка формирования сигнала, имитирующего паразитную помеху на передатчике вследствие прохожения сигнала передатчика через усилитель мощности.
\begin{figure}
	\centering
	\includegraphics[scale=0.35]{figures/install/install.pdf}
	\caption{Установка формирования второй гармоники сигнала передатчика: (a) Цифровой генератор сигналов; (b) Аналоговый усилитель мощности; (b) Цифровой осциллограф; (d) Компьютер.}
	\label{fig:install}
\end{figure}

Сигнал передатчика $\bit{x}$ записывается в генератор сигналов (рис. \ref{fig:install}). Длительность сигнала 2457600 отсчётов, что во времени соответствует $20$ мс. На вход радиочастотного усилителя мощности поступает сигнал передатчика в аналоговом виде на несущей частоте $1.7$ ГГц.

Цифровой осциллограф запрограммирован на выделение второй гармоники сигнала на частоте $3.4$ ГГц на выходе радиочастотного усилителя мощности. Частота дискретизации АЦП осциллографа $f_s=122.88$~МГц выбрана в соответствии со стандартом 5G \cite{3gpp_sample_rate}. Обработка сигнала помехи ведётся на компьютере в системе Python. 

Выделение второй гармоники $\bit{f}$ из потока цифрового сигнала происходит путём подсчета взаимной корреляции сигнала $\bit{f}$ и квадрата сигнала передатчика~$\bit{x}^2$~, как изображено на рис.~\ref{fig:capture}. Последующие 2457600 отсчётов сигнала $\bit{f}$, начиная с отсчёта, при котором зафиксирован максимум взаимной корреляции, считаются копией сигнала второй гармоники.
\begin{figure}
	\centering
	\includegraphics[scale=0.8]{figures/capture/capture.pdf}
	\caption{Модуль функции взаимной корреляции сигнала второй гармоники $\bit{f}$ и квадрата сигнала передатчика $\bit{x}^2$}
	\label{fig:capture}
\end{figure}

В результате имеются два массива длиной по 2457600 отсчётов каждый: сигнал передатичка $\bit{x}$ и сигнал второй гармоники $\bit{f}$. Спектральные плотности мощности этих сигналов изображены на рис. ~\ref{fig:tx}, ~\ref{fig:pa_out}.
\begin{figure}
	\begin{subfigure}[b]{0.51\textwidth}
		\includegraphics[scale=0.45]{figures/signal_generation/tx_rb75.pdf}
		\caption{}
		\label{fig:tx}
	\end{subfigure}
	\hspace{2ex}
	\begin{subfigure}[b]{0.51\textwidth}
		\includegraphics[scale=0.45]{figures/signal_generation/pa_out_rb75.pdf}
		\caption{}
		\label{fig:pa_out}
	\end{subfigure}
	\caption{Спектральная плотность мощности: (\subref{fig:tx}) сигнал, формируемый на передатчике для отправки в канал связи $\bit{x}$; (\subref{fig:pa_out}) вторая гармоника $\bit{f}$, образованная после прохождения сигнала передатчика через усилитель мощности }
	\label{fig:tx_pa_out}
\end{figure}

Для симуляции прохождения второй гармоники по различным путям распространения от передатчика к приёмнику производится свёртка массива отсчётов $\bit{f}$ с коэффициентами $\bit{w}$ импульсной характеристики КИХ-фильтра:
\begin{equation}
	\bit{d}=\bit{f}*\bit{w}
	\label{rx_noise_generate}
\end{equation}

Таким образом, получаем сигнал \bit{d}, имитирующий сигнал паразитной второй гармоники сигнала передатчика на приёмнике мобильного терминала. 

Импульсная характеристика \bit{w} формируется в системе Python в соответствии с данными о реальном канале распространения помехи.

Частотная характеристика канала распространения помехи от передатчика к приёмнику изображена на рис.~\ref{fig:firfr0}, спектральная плотность мощности сигнала паразитной помехи~\bit{d} изображена на рис.~\ref{fig:rx}.
\begin{figure}
	\begin{subfigure}[b]{0.51\textwidth}
		\includegraphics[scale=0.45]{figures/freq_resp/path2.pdf}
		\caption{}
		\label{fig:firfr0}
	\end{subfigure}
	\hspace{2ex}
	\begin{subfigure}[b]{0.51\textwidth}
		\includegraphics[scale=0.45]{figures/signal_generation/rx_rb75_path2.pdf}
		\caption{}
		\label{fig:rx}
	\end{subfigure}
	\caption{(\subref{fig:tx}) Амплитудно-частотная характеристика канала распространения второй гармоники от передатчика к приёмнику; (\subref{fig:pa_out}) паразитная помеха на приёмнике \bit{d}}
	\label{fig:firfir_example}
\end{figure}