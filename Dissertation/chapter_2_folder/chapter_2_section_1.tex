\section{Структура тестовой платформы} \label{sec:testbench_descr}

В данном разделе описывается тестовая установка, на которой сформирован сигнал, имитирующий нелинейно искаженный сигнал усилителем мощности в передатчике устройства связи. 

Измерительная установка состоит из персонального компьютера (ПК), генератора сигналов (ГС) R\&S SMW200A, анализатора спектра (АС) R\&S FSW85 и испытуемого усилителя мощности ZKY66291-11.

ПК загружает синфазную и квадратурную составляющие базовой полосы (IQ-данные) в ГС, который осуществляет модуляцию, переносит сигнал на несущую частоту $f_{\text{гет}}=1,8$~ГГц и передает сигнал на вход УМ. Выходной сигнал УМ с нелинейными искажениями поступает на АЦП анализатора спектра с частотой дискретизации $f_s=245.76$~МГц, выбранной в соответствии со стандартом 5G \cite{3gpp_sample_rate}. После этого IQ-данные базовой полосы передаются в ПК для последующей обработки сигнала на языке программирования Python. 

Регулировка входной мощности УМ осуществлялась с помощью ГС. В экспериментах использовался комплексный 20-МГц ортогональный сигнал с частотным мультиплексированнием (OFDM) сигнал со 100 ресурсными блоками.
\begin{figure}[h!]
	\centering
	\includegraphics[scale=1.0]{figures/testbench/testbench.pdf}
	\caption{Установка формирования нелинейных искажений в передатчике устройства связи.}
	\label{fig:install}
\end{figure}

Выделение нелинейно искаженного сигнала на выходе УМ $\text{PA(\bit{x})}$ из потока цифрового сигнала происходит путём подсчета взаимной корреляции сигнала $\text{PA(\bit{x})}$ и нелинейного терма передатчика~$\bit{x}\odot|\bit{x}|$, как изображено на рис.~\ref{fig:capture}, где~$\bit{x}\odot|\bit{x}|$ -- вектор $j$-ым элементом которого является отсчет $x_{n-j}|x_{n-j}|$.
\begin{figure}[h!]
	\centering
	\includegraphics[scale=0.7]{figures/capture/capture.pdf}
	\caption{Модуль функции взаимной корреляции сигнала на выходе УМ $\text{PA(\bit{x})}$ и нелинейного терма~$\bit{x}\odot|\bit{x}|$.}
	\label{fig:capture}
\end{figure}

Таким образом получены сигнал передатичка $\bit{x}$ и сигнал на выходе УМ $\text{PA(\bit{x})}$. Спектральные плотности мощности этих сигналов изображены на рис.~\ref{fig:tx},~\ref{fig:pa_out}.
\begin{figure}
	\begin{subfigure}[b]{0.51\textwidth}
		\includegraphics[scale=0.42]{figures/data_for_thesis/IBFD_single_band/tx.pdf}
		\caption{}
		\label{fig:tx}
	\end{subfigure}
	\hspace{2ex}
	\begin{subfigure}[b]{0.51\textwidth}
		\includegraphics[scale=0.42]{figures/data_for_thesis/IBFD_single_band/pa_out.pdf}
		\caption{}
		\label{fig:pa_out}
	\end{subfigure}
	\caption{Спектральная плотность мощности: (\subref{fig:tx}) сигнал, формируемый на передатчике для отправки в канал связи $\bit{x}$; (\subref{fig:pa_out}) сигнал $\text{PA(\bit{x})}$, образованный после прохождения сигнала передатчика через усилитель мощности}
	\label{fig:tx_pa_out}
\end{figure}