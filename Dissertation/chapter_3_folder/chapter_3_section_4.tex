\section{Методы экономии ресурсов при реализации нейросетевых моделей компенсации нелинейных искажений}

\subsection{Методы экономии ресурсов при реализации цифровых фильтров}

В предыдущих разделах была рассмотрена компенсация паразитной помехи в приёмнике мобильного устройства путём адаптациии модели Гаммерштейна в реальном времени. Стохастические алгоритмы позволяют потактово обновлять коэффициенты модели.

Предположим, что импульсный отклик канала распространения паразитной помехи от передатчика к приёмнику меняется незначительно в процессе работы мобильного устройства. В таком случае коэффициенты КИХ-фильтра можно зафиксировать после адаптации, а в реальном времени будут адаптироваться только коэффициенты слоя нелинейности амплитудной характеристики усилителя.

Поскольку в таком случае коэффициенты фильтра могут быть зафиксированы после адаптации, то есть возможность организовать экономию таких ресурсов, как мощность потребляемая цифровой схемой, а также площадь, занимаемая фильтром на кристалле RF-чипсета.

В эксперементах, приведённых в предыдущих разделах, использовались длинные КИХ-фильтры порядка $M=38$ и выше, что соответствует аппаратной реализации на кристалле RF-чипсета не менее 39 умножителей. 

Реализация умножителей является наиболее трудной с точки зрения используемых ресурсов по сравнению с другими элементами прямой формы реализации КИХ-фильтра, то есть сумматорами и элементами задержки. Это связано с тем, что мощность, потребляемая умножителем, а также площадь, занимаемая на кристалле, растут квадратично в зависимости от разрядности умножителя, в то время как потребление этих же ресурсов сумматорами и элементами задержки растёт линейно в зависимости от разрядности.

В связи с этим предлагается использовать метод квантования коэффициентов  КИХ-фильтра, позволяющий сократить количество используемых умножителей за счёт выражения одних коэффициентов фильтра через произведение других коэффициентов и степеней двойки. 

С точки зрения аппаратной реализации умножители заменяются на сумматоры и битовые сдвиги. Приведём пример из статьи \cite{fir_quant}. Пусть имеется КИХ-фильтр, реализованный в прямой форме, со следующими коэффициентами:
\begin{equation}
	\bit{h}=\begin{pmatrix}
		h_0 & h_1 & h_2 & h_3 & h_4
	\end{pmatrix}^T=
	\begin{pmatrix}
		1 & 2 & 4 & 3 & 6
	\end{pmatrix}^T.
	\label{quant_fir_example}
\end{equation}
В данном примере коэффициенты $h_0$, $h_1$, $h_3$ точно выражаются через коэффициенты $h_2$, $h_4$, однако в общем случае такой возможности нет. Алгоритм, приведённый в статье \cite{fir_quant} формирует импульсную характеристику, соответствующую уменьшеному числу умножителей таким образом, чтобы минимизировать отклонение исходных коэффициентов от тех, что формируется алгоритмом. При этом желаемое количество умножителей задаётся пользователем.

Приведём примеры работы алгоритма для сокращения числа умножителей линейно-фазового Фильтра Нижних Частот, линейно-фазового Полосового Фильтра, а также минимально фазового Фильтра Нижних Частот.

Рассмотрим линейно-фазовый Фильтр Нижних Частот порядка $N=138$. Пусть коэффициенты его импульсной характеристики $\bit{h}=\{h_n\}, n\in\overline{0, N-1}$. Пусть при этом желаемое количество умножителей $M=25$. Импульсную характеристику сформированного алгоритмом фильтра обозначим $\bit{g}=\{g_n\}, n\in\overline{0, N-1}$.

На рис. \ref{fig:quant_lin_phase_lpf_ir} изображена импульсная характеристика исходного фильтра нижних частот $h_n$, а также отклонение исходных коэффициентов от коэффициентов фильтра, упрощенного с точки зрения аппаратной реализации $h_n-g_n$. Это отклонение мало и составляет порядка $10^{-3}$.
\begin{figure}[h!]
	\centering
	\includegraphics[scale=1.0]{figures/linear_phase_lpf/LPF_ir.pdf}
	\caption{Линейно-фазовый Фильтр Нижних Частот. Исходная импульсная характеристика $h_n$ и отклонение $h_n-g_n$}
	\label{fig:quant_lin_phase_lpf_ir}
\end{figure}

На рис. \ref{fig:quant_lin_phase_lpf_fr} изображена частотная характеристика исходного фильтра $H(\omega)$, фильтра, построенного на 25 умножителях $G(\omega)$, а также разность частотных характеристик $E(\omega)=H(\omega)-G(\omega)$.

Из рис. \ref{fig:quant_lin_phase_lpf_fr} видно, что величина отклонения частотной характеристики фильтра, построенного на 25 умножителях от частотной характеристики исходного фильтра порядка $N=138$, реализованного в прямой форме, составляет -60 dB.
\begin{figure}[h!]
	\centering
	\includegraphics[scale=1.0]{figures/linear_phase_lpf/LPF_spectrum.pdf}
	\caption{Линейно-фазовый Фильтр Нижних Частот. Исходная частотная характеристика $H(\omega)$, частотная характеристика $G(\omega)$ и отклонение $E(\omega)$}
	\label{fig:quant_lin_phase_lpf_fr}
\end{figure}

Рассмотрим линейно-фазовый Полосовой Фильтр порядка $N=138$. Пусть коэффициенты его импульсной характеристики $\bit{h}=\{h_n\}, n\in\overline{0, N-1}$. Пусть при этом желаемое количество умножителей $M=25$. Импульсную характеристику сформированного алгоритмом фильтра также обозначим $\bit{g}=\{g_n\}, n\in\overline{0, N-1}$.

На рис. \ref{fig:quant_lin_phase_bpf_ir} изображена импульсная характеристика исходного полосового фильтра $h_n$, а также отклонение исходных коэффициентов от коэффициентов фильтра, упрощенного с точки зрения аппаратной реализации $h_n-g_n$. Это отклонение мало и составляет порядка $10^{-5}$.
\begin{figure}[h!]
	\centering
	\includegraphics[scale=1.0]{figures/linear_phase_bpf/BPF_ir.pdf}
	\caption{Линейно-фазовый Полосовой Фильтр. Исходная импульсная характеристика $h_n$ и отклонение $h_n-g_n$}
	\label{fig:quant_lin_phase_bpf_ir}
\end{figure}

На рис. \ref{fig:quant_lin_phase_bpf_fr} изображена частотная характеристика исходного фильтра $H(\omega)$, фильтра, построенного на 25 умножителях $G(\omega)$, а также разность частотных характеристик $E(\omega)=H(\omega)-G(\omega)$.

Как видно из рис. \ref{fig:quant_lin_phase_bpf_fr} величина отклонения частотной характеристики фильтра, построенного на 25 умножителях от частотной характеристики исходного фильтра порядка $N=138$, реализованного в прямой форме, составляет, так же как и для линейно-фазового ФНЧ, -60 dB.
\begin{figure}[h!]
	\centering
	\includegraphics[scale=1.0]{figures/linear_phase_bpf/BPF_spectrum.pdf}
	\caption{Линейно-фазовый Полосовой Фильтр. Исходная частотная характеристика $H(\omega)$, частотная характеристика $G(\omega)$ и отклонение $E(\omega)$}
	\label{fig:quant_lin_phase_bpf_fr}
\end{figure}

Рассмотрим минимально фазовый Фильтр Нижних Частот порядка $N=138$. Пусть коэффициенты его импульсной характеристики $\bit{h}=\{h_n\}, n\in\overline{0, N-1}$. $\bit{g}=\{g_n\}, n\in\overline{0, N-1}$ -- импульсная характеристика сформированного алгоритмом фильтра. Рассмотрим случай уменьшения количества умножителей до $M=25$ и до $M=70$.

Данный пример интересен в связи с тем, что в данном случае импульсная характеристика исходного фильтра $\bit{h}=\{h_n\}$ не является симметричной.

На рис. \ref{fig:quant_min_phase_lpf_ir} изображена импульсная характеристика исходного фильтра нижних частот $h_n$ и отклонение $h_n-g_n$ исходных коэффициентов от коэффициентов фильтра, построенного на 25 умножителях. Это отклонение составляет порядка $10^{-4}$.

\begin{figure}[h!]
	\centering
	\includegraphics[scale=1.0]{figures/minimum_phase_lpf/min_phase_LPF_ir.pdf}
	\caption{Минимально фазовый Фильтр Нижних Частот. Исходная импульсная характеристика $h_n$ и отклонение $h_n-g_n$}
	\label{fig:quant_min_phase_lpf_ir}
\end{figure}
На рис. \ref{fig:quant_min_phase_lpf_fr} изображена частотная характеристика исходного фильтра $H(\omega)$, фильтра, построенного на 25 и 70 умножителях $G(\omega)$, а также разность частотных характеристик $E(\omega)=H(\omega)-G(\omega)$ для обоих случаев.

Из рис. \ref{fig:quant_min_phase_lpf_fr} видно, что величина отклонения частотной характеристики фильтра, построенного на 25 умножителях от частотной характеристики исходного фильтра порядка $N=138$, реализованного в прямой форме, составляет -50 dB. 

При этом отклонения частотной характеристики фильтра, построенного на 70 умножителях от частотной характеристики исходного фильтра порядка $N=138$ составляет -70 dB.

\begin{figure}[h!]
	\centering
	\includegraphics[scale=1.0]{figures/minimum_phase_lpf/min_phase_LPF_spectrum.pdf}
	\caption{Минимально фазовый Фильтр Нижних Частот. Исходная частотная характеристика $H(\omega)$, частотная характеристика $G(\omega)$ и отклонение $E(\omega)$}
	\label{fig:quant_min_phase_lpf_fr}
\end{figure}
Таким образом, алгоритм, приведённый в статье \cite{fir_quant}, позволяет существенно сократить мощность потребляемую цифровым КИХ-фильтром, а также занимаемую площадь кристалла RF-чипсета, за счёт замены части умножителей более простыми с точки зрения аппаратной реализации и затрачиваемых ресурсов битовыми сдвигами и дополнительными сумматорами.

\subsection{Сокращение ресурсов при реализации свёрточных сетей для задачи компенсации нелинейных искажений}