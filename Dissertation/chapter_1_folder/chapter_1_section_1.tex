\section{Постановка задачи адаптивной компенсации нелинейных искажений в устройствах связи}

\subsection{Проблема возникновения паразитных помех в приёмном тракте приемо-передающего устройства}
В данном разделе формулируется задача адаптивной компенсации нелинейных паразитных помех, возникающих на приёмнике приемо-передающего устройства. Приводится схема адаптивной компенсации данного рода помех. А также описывается модель формирования паразитных помех на основе качественного представления физических процессов.

\label{sec:parasit_interfere}
В качестве примера возникновения паразитных помех в приёмном тракте рассмотрим приёмо-передающее устройство, состоящее из двух каналов передатчика и приёмника (2T2R), как это показано на рис. \ref{fig:2T2R_interf_prop}. При этом передатчик и приёмник разнесены по частоте. 

В процессе прохождения сигнала на несущей частоте $f_0$ через нелинейные цепи передатчика, генерируются компоненты на кратных частотах $f_0$, $2f_0$, $3f_0$ и выше. Гармоники порядка выше 3-его, как правило, имеют существенно меньшую мощность по сравнению с гармониками 1-го, 2-го и 3-его порядка. 

Среди элементов передающего тракта, представленных на рис. \ref{fig:2T2R_interf_prop}, наибольших вклад в формирование нелинейных компонент спектра вносит аналоговый усилитель мощности. В связи с этим будем далее считать, что именно усилитель мощности является источником нелинейности. Согласно стандарту LTE, используемом в системах связи четвёртого поколения 4G, пара передатчик-приёмник может работать на таких частотах, при которых первая, вторая или третья гармоника сигнала передатчика попадает в полосу приёмного тракта того же устройства \cite{3gpp_36_104}: 
\begin{itemize}
	\item LTE Band 2 (UL 1920–1980 MHz) – LTE Band 2 (DL 1930–1990 MHz)
	\item LTE Band 8 (UL 880-915 MHz) - LTE Band 3 (DL 1805-1880 MHz)
	\item LTE Band 17 (UL 704-716 MHz) - LTE Band 4 (DL 2110-2155 MHz)
\end{itemize}
Кроме того, за счёт малых размеров RF-чипсетов, компоненты внутри чипсета расположены компактно. Вследствие технологических ограничений производства существует сложность обеспечения изоляции отдельных компонент RF-чипсета. В связи с этим появляются различные пути распространения помехи в приёмнике, как это показано на рис. \ref{fig:2T2R_interf_prop}.
\begin{figure}[htbp]
	\centering
	\includegraphics[scale=0.4]{figures/2t2r_interf_propagate_1_harmonic/2t2r_interf_propagate_1_harmonic.pdf}  
	\caption{Каналы распространения 1-ой гармоники нелинейных помех в приёмо-передающем устройстве 2T2R}
	\label{fig:2T2R_interf_prop}
\end{figure}

Пример спектра сигнала передатчика на несущей частоте $f_0=1.93$~ГГц и помехи, возникающей в приёмнике на несущей частоте передатчика $f_1=1.96$~ГГц изображен на рис. \ref{fig:first_harmonic}. Задача модуля компенсации заключается в том, чтобы понизить уровень этой помехи для обеспечения требуемого уровня чувствительности приёмника.
\begin{figure}[htbp]
	\centering
	\includegraphics[scale=0.8]{figures/interf_example/first_harmonic/first_harmonic.pdf}
	\caption{Спектральная плотность мощности сигнала передатчика на выходе нелинейного УМ на частоте $f_0=1.93$ ГГц и полезного сигнала приёмника на частоте $f_1=1.96$ ГГц}
	\label{fig:first_harmonic}
\end{figure}

Таким образом, одной из задач данной работы является разработка и исследование модуля адаптивной нелинейной коррекции, который позволит компенсировать паразитные помехи в приёмника. Кроме того, такой подход позволит снизить требования к блокам фильтрации аналогового сигнала: дуплексерам и полосовым фильтрам.

\subsection{Схема компенсации паразитных нелинейных помех в приёмном тракте мобильного терминала} \label{subsec:sic_describtion}
\label{sec:problem_form}
Как отмечалось ранее, будем считать, что основным источником нелинейных помех в приёмном тракте является аналоговый усилитель мощности в передающем тракте мобильного терминала \cite{dig_front_end}.

Предлагается идентифицировать помехи, попадающих в приёмник путем адаптации нелинейной модели по критерию минимума среднего квадрата ошибки \cite{adapt_filt}. Такая схема компенсации нелинейных помех изображена на рис. \ref{fig:ident_problem}.
\begin{figure}
	\centering
	\includegraphics[scale=0.6]{figures/ident_problem/ident_problem.pdf}
	\caption{Схема компенсации нелинейных помех в приёмнике приёмо-передающего устройства}
	\label{fig:ident_problem}
\end{figure}

Исходный цифровой сигнал $x$ на несущей частоте $f_0$ проходит через блок нелинейной модели $f(\cdot)$, в результате чего на её выходе формируется цифровой сигнал $y=f(x)$. Cигнал проходит по передающему тракту: через ЦАП, RF-модуль, аналоговый усилитель мощности, затем вследствие наличие различных путей распространения, паразитные гармоники исходного сигнала на частотах $f_0$, $2f_0$, $3f_0$ попадают на приёмную часть. 

Проходя по приёмному тракту помеха приобретает вид цифрового сигнала, обозначенного как $d$. Помимо помехи в приёмную часть терминала поступает полезный сигнал $r$. Таким образом, на выходе АЦП -- сумма $r+d$. 

Блок адаптации $АД$ производит идентификацию, подстраивая коэффициенты $z$ нелинейной модели $f(\cdot)$ посредством измерения отклонения $e=d-y$ принятой приёмным трактом помехи от выхода нелинейной модели. В общем случае можно считать, что коэффициент корреляции отправляемого на передатчике сигнала $x$ и полезного принятого сигнала $r$ равен нулю. Поэтому в результате работы модуля компенсации результирующий сигнал на приёмнике будет стремиться к полезному принятому сигналу $r+e\rightarrow r$. Задачу идентификации паразитной помехи путём адаптации нелинейной модели по критерию минимума среднего квадрата сформулируем следующим образом:
\begin{equation}
	\begin{cases}
		J = e^*e\rightarrow \displaystyle \min_{z},\\
		e = d-f(x, z) 
	\end{cases}
\end{equation}
%\subsection{Аппроксимация паразитных нелинейных характеристик аналоговых элементов и линейного канала распространения помехи из передатчика в приёмный тракт мобильного терминала}
%\label{sec:nonlin_model}
%Как отмечалось ранее в передатчике сигнал проходит через нелинейные элементы, такие как усилители мощности и дуплексеры, в результате чего формируются гармоники на частотах кратных несущей частоте сигнала передатчика. Модель паразитных помех $NL$ (рис. \ref{fig:ident_problem}) должна учитывать подобного рода нелинейные искажения. 
%
%Помимо этого, модель должна учитывать, тот факт что в результате компактного расположения компонент, а также плохой изоляции отдельных компонент чипсета, как отмечалось ранее, появляются множественные пути распространения помехи из передатчика в приёмник (рис. \ref{fig:2T2R_interf_prop}). Причём различные копии помехи имеют различные задержки. 
%
%Подобного рода помехи описываются при помощи модели Гаммерштейна \cite{pa_models}, представленной на рис.~\ref{fig:hammerstein}, в соответствии с физическими процессами образования и распространения помехи. 
%
%В модели присутствует КИХ-фильтр, наличие которого обосновывается наличием канала распространения помехи из передатчика в приёмник. КИХ-фильтр описывается коэффицинтами \bit{w}. Порядок фильтра определяется величиной задержек помехи при её распространении внутри чипа.
%
%В модели присутствует нелинейный слой $g(\cdot)$, описываемый коэффициентами \bit{h}. Нелинейный слой является безынерционным, поскольку инерционные составляющие, в том числе составляющие аналоговых нелинейных компонент, учитывается КИХ-фильтром.
%
%Отметим, что блок $g(\cdot)$ описывает нелинейность амплитудной характеристики усилителя мощности, поэтому данный блок является функцией вещественного аргумента. 
%
%Поскольку ведётся обработка комплексной огибающей \cite{dsp_layons}, коэффициенты \bit{h}, описывающие функцию $g(\cdot)$, должны принимать комплексные значения. Таким образом, $g(\cdot)$ -- комплекснозначная функция вещественного аргумента.
%
%Выход нелинейного блока $g(\cdot)$ умножается на входной сигнал в степени $q$. Значение степени $q$ выбирается в зависимости от порядка паразитной гармоники. К примеру если стоит задача скомпенсировать паразитную гармонику второго порядка, как это показано на рис. \ref{fig:second_harmonic}, тогда $q=2$.
%\begin{figure}
%	\centering
%	\includegraphics[scale=0.7]{figures/hammerstein/hammerstein.pdf}
%	\caption{Модель Гаммерштейна}
%	\label{fig:hammerstein}
%\end{figure}
%
%Пусть на входе модели Гаммерштейна -- отсчёт $x_n$, на входе модели нелинейности $g(\cdot)$ -- модуль входного сигнала $|x_n|$, на входе фильтра -- $z_n$, выход модели -- $y_n$. Выход нелинейного блока описывается выражением:
%\begin{equation}
%	z_n=g(|x_n|)x_n^q
%	\label{fir_input}
%\end{equation}
%
%Обозначим порядок КИХ-фильтра как $M$. В результате, на выходе модели Гаммерштейна будет цифровой сигнал следующего вида:
%\begin{equation}
%	y_n=\displaystyle\sum_{k=0}^{M} w_k x_{n-k}^qg(|x_{n-k}|).
%	\label{hammerstein_out_scalar}
%\end{equation}

\subsection{Проблема возникновения нелинейных искажений сигнала передающего тракта}
Ввиду нелинейности харакеристики передающего тракта формируются нелинейные искажения, которые значительно влияют на качество радиочастотных сигналов, создавая помехи в канале связи. Такие нелинейные искажения приводят к увеличению битовой ошибки на приемнике, генерируют внутриполосные и внеполосные помеховые сигналы и ухудшают качество передачи сигналов соседних полос. 

Как было отмечено ранее, наибольших вклад в искажения сигнала в передающем тракте вносят нелинейные усилители мощности. В результате прохождения через УМ формируются компонентны спектра внутри полосы передатчика, а также внеполосные искажения. На рис.~\ref{fig:tx_digial_distortion} изображена спектральная плотность мощности сигнала передатчика до и после прохождения через УМ.

\begin{figure}[ht]
	\centering
	\includegraphics[scale=0.8]{figures/tx_digial_distortion/tx_digial_distortion.pdf}
	\caption{Спектральная плотность мощности сигнала передатчика  на входе и выходе усилителя мощности на частоте $f_0=1.93$ ГГц}
	\label{fig:tx_digial_distortion}
\end{figure}

\subsection{Схема компенсации нелинейных искажений в передатчике приемо-передающего устройства} \label{subsec:dpd_describtion}

Для предотвращения искажений в современных базовых станциях и сотовых устройствах широко используются методы цифровой предискажения DPD [1]-[3].

Устройство DPD представлено блоком с обратной нелинейной характеристикой, изменяющим входной сигнал УМ так, чтобы минимизировать нелинейных искажений на выходе УМ (рис.~\ref{fig:dpd_structure}).
\begin{figure}
	\centering
	\includegraphics[scale=0.6]{figures/dpd_dynamic/dpd_scheme/dpd_scheme.pdf}
	\caption{Структура системы цифрового предыскажения}
	\label{fig:dpd_structure}
\end{figure}

Таким образом, DPD предыскажает входной сигнал УМ таким образом, чтобы на выходе сформировался линейно искаженный сигнал. Такой подход позволяет работать в режимах высокой нелинейности УМ и, в результате, повысить КПД УМ. 
Задача оптимизации цифрового предискажения может быть представлена математическим выражением [4], описывающим минимизацию отклонения входа DPD от выхода УМ:
\begin{equation}
	\dsp||g(x - f(x, z)) - x||_2^2 \rightarrow \dsp\min_{z},
	\label{dpd_task_general}
\end{equation}
где $x$ -- входной сигнала УМ, $g(\cdot)$ -- нелинейная характеристика УМ, $f(\cdot)$ -- модель цифрового предыскажения с адаптивными параметрами $z$.
Предположим, что нелинейную характеристику УМ $g(x)$  можно аппроксимировать линейной функцией в окрестности рабочей точки:
\begin{equation}
	\dsp g(x - f(x, z)) \approx g(x) - g'_x(x)f(x, z),
	\label{dpd_decomp}
\end{equation}
где $g'_x(x)$ -- производная нелинейной функции УМ по входу. Поскольку DPD должен работать в таком режиме УМ, чтобы уровень нелинейных искажений оставался значительно ниже уровня передаваемого сигнала, то $g'_x(x)\approx I$ -- будет представлять собой единичную матрицу. Подставляя~\eqref{dpd_decomp} в \eqref{dpd_task_general}, получаем выражение:
\begin{equation}
	\dsp||f(x, z) - e||_2^2 \rightarrow \dsp\min_{z},
	\label{dpd_task_simplif}
\end{equation}
где $e=g(x) - x$ -- вектор ошибки.

%\subsection{Критерий адаптации паразитных нелинейных помех мобильного терминала}
%Ввиду простоты аппаратной реализации модели кусочно-линейной аппроксимации в качестве модели нелинейностей аналоговых компонент, в дальнейшем будем считать, что нелинейность $g(\cdot)$ в модели Гаммерштейна (рис. \ref{fig:hammerstein}) описывается блоком кусочном-линейной аппроксимации PLA.
%
%Существует несколько способов обработки сигнала \cite{adapt_filt_haykin}. Cимуляция алгоритмов как правило проводится на сохраненных блоках данных. В этом случае имеется мощный вычислитель (компьютер) и возможность вести математическую обработку матриц и других объектов алгоритмов. Это блочный режим работы. При таком методе обработки сигнала корреляционная матрица сигнала и матрица блока PLA, введённая по аналогии с корреляционной матрицей сигнала, вычисляются по формулам \eqref{autorokk_matrprod}, \eqref{autorokk_matrprod_pla} соответственно. 
%
%При аппаратной реализации блочный режим стараются заменить стохастическим в реальном времени \cite{adapt_filt_haykin}, чтобы минимизировать задержки и сократить ресурсы. В этом случае векторы состояния, корреляционные матрицы, шаг алгоритма, а также другие объекты, обновляются каждый отсчет. Корреляционная матрица сигнала и матрица корреляции соответствующая блоку PLA вычисляются по формулам \eqref{autokorr_expect}, \eqref{autokorr_expect_pla} соответственно.
%
%В последующих разделах будем рассматривать алгоритмы адаптации КИХ-фильтра, блока PLA и модели Гаммерштейна.
%
%При рассмотрении методов компенсации паразитных помех будем считать, что на вход блока адаптации поступает вектор отсчетов передатчика $\textbf{\textit{x}}$ длины $N$, вектор отсчетов с приёмника $\textbf{\textit{d}}$ также длины $N$.
%
%Задача блока адаптации заключается в том, чтобы так преобразовать исходный вектор $\textbf{\textit{x}}$, чтобы как можно лучше приблизить к вектору отсчетов на приёмнике $\textbf{\textit{d}}$ путём минимизации нормы вектора ошибки $\textbf{\textit{e}}=\textbf{\textit{d}}-f(\textbf{\textit{x}})$, где $f(\cdot)$ -- оператор преобразования исходного вектора в вектор отсчетов на выходе модели помехи. Оператор $f(\cdot)$ определяется коэфициентами $\textbf{\textit{w}}, \textbf{\textit{h}}$.
%
%Рассмотрим требования к целевой функции (метрике) алгоритмов компенсации нелинейной модели парахитных помех. Метрика величины отклонения выхода модели помехи от вектора отсчетов самой помехи должна быть вещественнозначной и неотрицательной для реализации процедуры поиска минимума данной метрики. Кроме того, желательно задать целевую функцию квадратичной и выпуклой для применения эффективных методов адаптации \cite{convex_opt}.
%
%Такой вещественной, неотрицательной и квадратичной метрикой является средний квадрат ошибки (MSE - Mean Square Error). Физический смысл среднего квадрата ошибки -- энергия отклонения выхода нелинейной модели от помехи, измеренной на приёмнике. При блочной обработке сигнала MSE имеет вид:
%\begin{equation}
%	J=\textbf{\textit{e}}^H\textbf{\textit{e}}.
%	\label{mse_block}
%\end{equation}
%Помимо MSE на практике применяется метрика - нормированный средний квадрат ошибки (NMSE - Normalized Mean Square Error):
%\begin{equation}
%	J=\frac{\textbf{\textit{e}}^H\textbf{\textit{e}}}{\textbf{\textit{d}}^H\textbf{\textit{d}}}.
%	\label{nmse_block}
%\end{equation}
%В данном случае энергия ошибки нормируется к энергии сигнала приёмника для того чтобы оценить уровень ошибки независимо от динамического диапазона сигнала на входе применика. 
%
%Реальные сигналы, используемые в системах связи являются мощностными \cite{sklyar_dig_telecom}, то есть обладают бесконечной энергией на бесконечном промежутке времени, однако в уравнениях \eqref{mse_block}, \eqref{nmse_block} фигурирует энергия ошибки и сигнала помехи, рассчитанная на конечной длине блока.
%
%Критерии MSE и NMSE для блочных методов адаптации в случае, когда в качестве модели паразитных помех используется модель Гаммерштейна, имеют вид:
%\begin{equation}
%	J=\textbf{\textit{e}}^H\textbf{\textit{e}}=
%	(\bit{d}-\bit{U}\bit{w})^H
%	(\bit{d}-\bit{U}\bit{w})=
%	(\bit{d}-\bit{V}_f\bit{h})^H
%	(\bit{d}-\bit{V}_f\bit{h})=
%	\rightarrow\min_{\textbf{\textit{w}}, \textbf{\textit{h}}},
%	\label{mse_block_criter}
%\end{equation}
%\begin{equation}
%	J=\frac{\textbf{\textit{e}}^H\textbf{\textit{e}}}{\textbf{\textit{d}}^H\textbf{\textit{d}}}=
%	\frac{
%		(\bit{d}-\bit{U}\bit{w})^H
%		(\bit{d}-\bit{U}\bit{w})
%	}
%	{\bit{d}^H\bit{d}}=
%	\frac{
%		(\bit{d}-\bit{V}_f\bit{h})^H
%		(\bit{d}-\bit{V}_f\bit{h})
%	}
%	{\bit{d}^H\bit{d}}
%	\rightarrow\min_{\textbf{\textit{w}}, \textbf{\textit{h}}},
%	\label{nmse_block_criter}
%\end{equation}
%где $\bit{U}$ -- матрица состояния КИХ-фильтра в модели Гаммерштейна \eqref{fir_state_matr_hammerst}, а $\bit{V}_f$ -- модификация матрицы состояния блока нелинейности $g(\cdot)$ в модели Гаммерштейна. В матрице $\bit{V}_f$ строки определяются выражениями $\bit{w}^T\bit{Q}_n$ (рис. \ref{fig:submatrix_q}).